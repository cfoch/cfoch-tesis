\documentclass[a4paper,openright,12pt]{report}

\usepackage[spanish]{babel}
\usepackage[utf8]{inputenc}
\usepackage{graphicx}
\usepackage{apacite}

% Support older LaTeX versions.
\DeclareUnicodeCharacter{2060}{}
\begin{document}
\begin{titlepage}
	\centering

	{\sffamily\large\bfseries PONTIFICIA UNIVERSIDAD CATÓLICA DEL PERÚ \par}
	\vspace{0.2cm}
	{\sffamily\large\bfseries INGENIERÍA INFORMÁTICA\par}
	\vfill

	\includegraphics[width=6cm]{../images/logo-pucp.png}\par\vspace{1cm}
	\vspace{1.5cm}

	{\sffamily\large\bfseries Implementación de un plug-in para GStreamer para
	el seguimiento de rostros aplicado a Cheeese\par}
	\vspace{2cm}

	{\sffamily\small Tesis para optar por el título de Ingeniero Informático que presente el
	bachiller: }

	{\sffamily\Large\bf	César Fabián Orcccón Chipana \par}
	{\sffamily 20105515 \par}
	\vfill
	{\sffamily Asesor: Dr.~Ivan \textsc{Sipiran} \par}

	\vfill

% Bottom of the page
	{\sffamily\large \today\par}
\end{titlepage}



\tableofcontents
\chapter{Generalidades}
\section{Problemática}


Por más de 130 años las personas han usado cámaras para documentar sus vidas.
Cuando las primeras cámaras fueron inventadas, era difícil y costoso tomarse una
fotografía \cite{snap2017prospectus}. Un filtro fotográfico es un accesorio para las
cámaras fotográficas que se inserta en la parte frontal de esta para conseguir
cierto efecto deseado. La historia de los filtros fotográficos también es muy
antigua, y se remonta al siglo XIX. En 1878, Frederick Wratten, un innovador e
investigador del campo de la fotografía introdujo un proceso conocido como
“noodling” que permitía crear placas fotográficas más sensitivos que las
existentes en esa época. En 1906, él junto con su hijo y el doctor C. E. Kenneth
Mees fundaron una compañía llamada Wainwright Ltd., en la cual Mees ayudó a
Wratten a desarrollar las primeras placas pancromáticas y los primeros filtros
de luz, los cuales se harían conocidos como Wratten filters. Kodak compraría
más tarde Wainwright Ltd. en el 1912 \cite{hannavy2013encyclopedia}⁠. Desde
entonces, los filtros fotográficos han ganado popularidad, y renovándose hasta
la actualidad.\\

En los últimos años, la aparición de la fotografía digital ha revolucionado la
industria fotográfica. En la actualidad, es común contar si no con una cámara
digital, entonces al menos con una cámara web integrada al computador personal
portátil o al telefóno inteligente. El término “filtro” (el cual de aquí y en
adelante será entendido como “filtro digital”) también se ha mudado al contexto
digital y se ha introducido en el hablar del común de muchas personas. Las
razones por las cuales las personas aplican filtros a sus fotografías son
varias. Algunos aplican filtros para mejorar la estética regulando el brillo,
contraste y focalización. Otros desean aplicar efectos antiguos, por ejemplo
algunos colocan las fotografías en blanco y negro para concentrar la atención
en cierta textura de la toma y evitar que el espectador se distraiga en los
colores, otros simplemente desean darle un aspecto de antigüedad. Otras personas
simplemente desean cambiar los colores. Finalmente, algunos otros simplemente
desean darle un aspecto único y divertido a las fotografías. El único propósito
de estos últimos es el agregar características divertidas (a través de filtros)
que no pudieron ser capturadas originalmente por la cámara
\cite{bakhshi2015we}.\\

Hoy en día, diversas empresas se han apalancado del fácil acceso a la cámara web
por parte de los usuarios de teléfonos inteligentes destacando entre ellas
empresas como Skype Technologies S.A.R.L (Skype), Oath Inc. (Flickr),
Facebook Inc. (Facebook e Instagram) y Snap Inc. (Snapchat) han permitido la
edición de imágenes aplicando filtros de manera sencilla para sus usuarios. Por
ejemplo, la aplicación para móviles Snapchat se ha hecho muy popular por no solo
aplicar filtros que permiten regular los colores de la foto o video en tiempo
real, sino que permite detectar objetos capturados por la cámara del dispositivo
y seleccionar entre una diversa variedad de lentes que permite sobreponer
animaciones interactivas sobre la toma \cite{snap2017prospectus}⁠. Este tipo de
filtros se han popularizado especialmente entre “millennials”. Por ejemplo, para
Snapchat, los jóvenes de entre 18 y 24 años representan un 36% de sus usuarios
activos por día en Estados Unidos \cite{snap2017prospectus}⁠. De hecho, muchos
jóvenes adultos sostienen que este tipo de filtros resuelve problemas de
comunicación dado en ciertas redes sociales debido a que la sobreposición de
imágenes y texto puede clarificar el mensaje que ellos desean expresar en sus
fotografías \cite{vaterlaus2016snapchat}. El deseo por agregar stickers sobre
las fotografías, no se limita a occidente. De hecho, esta práctica ha sido y es
muy común en Japón desde la década de los noventas. En el año 1995, Altus, una
empresa japonesa, desarrolló un fotomatón (un quiosco para tomarse fotos,
generalmente insertando una moneda) que permitía agregar stickers sobre las
fotografías. Atlus patentó una máquina con el nombre de Purikura, término que se
volvería más adelante en un término del uso cotidiano de las personas en Japón
principalmente los adolescentes \cite{edwards2004photographs}. Recientemente,
estas máquinas también se han renovado incluyendo filtros digitales para el
agregado de “stickers” sobre las fotografías \cite{TheOrigi29}.\\

En las computadoras de escritorio, podemos dividir la situación actual de las
aplicaciones de webcam por el Sistema Operativo en los cuales están soportadas.
Por un lado, en macOS, la aplicación para capturar fotos y videos con la cámara
web por defecto es Photo Booth, la cual permite aplicar filtros en tiempo real,
así como detectar a una persona para reemplazar el fondo por un video
preestablecido o foto colocada manualmente por el usuario. Fun Booth es una
aplicación para macOS. desarrollada por Spoonjuice LLC, la cual permite detectar
un rostro humano frente a la cámara para colocar stickers sobre este. Por otro
lado, Windows 10 incluye una aplicación para la toma de fotos y grabación de
video, pero esta no dispone de filtros tan sofisticados como los de Photo Booth.
Sin embargo, existen aplicaciones de terceros con varias funcionalidades.
CyberLink YouCam, de CyberLink Corp. permite no solo detectar el rostro del
individuo capturado y sobreponer filtros sobre esto, sino que detecta las
emociones de las personas \cite{YouCam7A82}⁠. ManyCam que entre diversas
funcionalidades que ofrece, permite además aplicar filtros para colocar máscaras
y animaciones que rastrean el rostro de la persona ubicada frente a la webcam
\cite{Webcamso75}. Para sistemas operativos basados en GNU/Linux, existen
programas que permiten ajustar los colores de las capturas con la cámara web,
pero la aplicación, al parecer, más popular y a la vez con filtros más
sofisticados es Cheese, proyecto de la GNOME Foundation \cite{AppsChee13}.⁠\\

Cheese, a diferencia del resto de programas mencionados con licencia privativa,
es un software libre. El software libre a diferencia del software privativo,
según la Free Software Foundation, es el software que respeta la libertad de sus
usuarios y comunidad permitiéndoles tener el derecho de ejecutar, copiar,
distribuir, estudiar, modificar y mejorar el programa \cite{whatIsFreeSoftware}⁠.
Cheese está licenciado bajo la licencia GNU General Public License versión 2
(GPLv2) \cite{cheeseLicense}⁠.
El hecho de que este tenga una licencia libre es fundamental para el desarrollo
de este proyecto de fin de carrera, pues permite a otros estudiar su código
fuente, mejorarlo y modificarlo. Este software, además, ha sido financiado por
Alphabet Inc. en varias ocasiones. De hecho, este fue inicialmente desarrollado
en el 2007 por Daniel G. Siegel como parte de un proyecto del Google Summer of
Code (GSoC), programa ideado por Larry Page y Sergey Brin, fundadores de Google,
para difundir el software libre y de código abierto remunerando a estudiantes
por desarrollar en este tipo de proyectos \cite{gsoc1}.\\


La necesidad por agregar soporte para sobreposición de imágenes sobre el rostro
de las capturas de Cheese fue expresado en el 2010 ⁠\cite{Bug6279256} y, además
ya hubo intentos de agregar esta funcionalidad. En el GNOME Outreach Program for
Women (programa para mujeres similar al GSoC pero financiado por la GNOME
Foundation, ahora llamado Oureachy) del 2010, la participante Laura Elisa Lucas
Alday, trabajó en desarrollar un filtro (elemento) para GStreamer llamado
gstfaceoverlay, el cual permitía la adición de imágenes en formato svg sobre un
rostro detectado en una imagen \cite{faceoverlay}\cite{gopw1}. El
filtro fue aceptado por los desarrolladores GStreamer, y posteriormente agregado
a GNOME Video Effects, librería que contiene una lista de filtros o combinación
de filtros de GStreamer en archivos de texto y que son usados por Cheese para
aplicar los filtros. Sin embargo, este filtro fue posteriormente retirado de
GNOME Video Effects puesto que algunos archivos svg hacían lento (el “framerate”
caía) a Cheese \cite{Bug6641489}.\\


En setiembre 2012, GStreamer 1.0 fue lanzado, y desde esa fecha,
gstfaceoverlay no fue portado a la nueva versión de
GStreamer. Esto llevó a que el filtro fuera borrado de la librería el
21 de diciembre del 2016. Ante esto, el autor de este proyecto de fin de
carrera, propuso un parche para portar el filtro a las series GStreamer 1.x en
el 2016 que sería aceptado en GStreamer en el año 2017 \cite{Bug7691771}.
En vista de que era un problema el hecho de soportar solo
formatos svg, quien redacta también escribió un parche para soportar más formatos
de imagen usando el filtro gdkpixbufoverlay, el cual, sin embargo, no ha sido
revisado hasta el momento. Una vez portado el filtro, en un intento en probar
el filtro en Cheese, se observó que este solo funciona, como se mencionó
inicialmente, para detectar el rostro de una sola persona, en otras palabras,
si hay múltiples personas frente a la cámara web, este filtro solo detectaría a
una: la primera (en el orden de detección por OpenCV). Quien redacta también
propuso un parche para solucionar el problema. Además se agregó la funcionalidad
a gstfaceoverlay de soportar múltiples imágenes sobre múltiples rostros
\cite{Bug769176}, y es ahí donde el problema fundamental en este filtro:
gstfaceoveraly como filtro para Cheese no debió implementarse sobre la base de
gstfacedetect que solo detecta rostros en vez de rastrearlos. Usar gstfacedetect
como base es un error si se desea usar el plugin en Cheese, pues este aplica la
detección facial para cada cuadro (o “frame” en inglés). No obstante,
gstfacedetect no respeta un orden alguno, es decir que una persona etiquetada en
el primer cuadro con la imagen A podría ser etiquetada en el segundo marco con
la imagen B, y en el tercero otra vez con A o tal vez con C. Lo esperado sería
que si se sobrepuso la imagen A sobre el rostro de una persona, esta imagen A
se mantenga sobrepuesta en los siguientes cuadros.\\

Para conseguir que Cheese soporte filtros de sobreposición de imágenes sobre los
rostros de personas en tiempo real no es suficiente escribir un nuevo filtro
para GStreamer. Tampoco es suficiente, luego de escribir el filtro, agregar la
configuración del “pipeline” de GStreamer a GNOME Video Effects de tal modo que
Cheese pueda leerlo. Cheese debería implementar una interfaz gráfica en la cual
el usuario pueda seleccionar e importar imágenes que desea usar, de este modo no
se evita el uso de efectos estáticos predeterminados. Es preciso resaltar que
realizar tales cambios a Cheese no es fácil para nuevos colaboradores del
proyecto. Cheese no se ha actualizado recientemente, y una de las razones puede
ser que no existe muchas que sepan programar en Vala. Además Cheese usa Clutter,
librería de GNOME basada en OpenGL, para mostrar la salida de la captura y los
efectos aplicados en la pantalla. Ambos no disponen de mucha documentación
disponible pública, y por lo general, uno debe guiarse en base a otros programas
que han sido escritos usando tanto Clutter como Vala. Por otro lado, entre los
filtros a usar para el rastreamiento de imágenes, algunos como face4d no tienen
licencias compatibles con GPLv2, para lo cual no solo es necesario escribir
software libre, sino que el código debe ser compatible con esta licencia.\\

En conclusión, se puede observar que los usuarios de software de webcam tienen
varias razones para preferir usar filtros sobre, sin embargo, las aplicaciones
de software libre para escritorio que capturan imagen y video desde la cámara
web están quedándose atrasadas tecnológicamente. Sin embargo, no solo son estas,
sino que pareciera que las aplicaciones para macOS también están en la misma
situación. Recientemente, la aparición de Snapchat ha creado una tendencia entre
millennials de 18 a 24 años a tener una preferencia por filtros que sobreponen
imágenes que siguen sus rostros capturados por la cámara web a otros medios
de comunicación convencionales. Siendo Cheese un software libre, lo cual
representa una ventaja social sobre el software privativo, es posible estudiarlo
y mejorarlo (sin mencionar otras libertades). Agregar la funcionalidad de
aplicar filtros que sobrepongan imágenes sobre los rostros de las personas
capturados por una webcam en tiempo real no solo es beneficio para los jóvenes
millennials, sino también para la GNOME Foundation y podría aumentar la
competitividad entre aplicaciones de escritorio de captura de video y entre
entornos de escritorio para sistemas basados en UNIX (incluyendo macOS).





\section{Objetivos}
\subsection{Objetivo general}
\subsection{Objetivos específicos}
\subsection{Resultados esperados}
\section{Herramientas y métodos}
\section{Alcance y limitaciones}
\section{Viabilidad}
\subsection{Viabilidad técnica}
\subsection{Viabilidad temporal}
\subsection{Viabilidad económica}
\chapter{Marco Conceptual}
\chapter{Estado del Arte}
\section{Revisión y conclusión}
\section{Conclusiones}
\chapter{Presentación de los resultados esperados}
\chapter{Conclusiones y trabajos futuros}
\section{Conclusiones}
\subsection{Trabajos futuros}

\bibliographystyle{apacite}
\bibliography{references}{}


\end{document}
